\documentclass[10pt]{article}
\usepackage[T1]{fontenc}
\usepackage{amssymb}
\usepackage{amsmath}
\usepackage{graphicx}
\usepackage{algpseudocode}
\usepackage{algorithm}
\usepackage{enumitem}


\usepackage{tikz}
\usetikzlibrary{arrows}
\usepackage{subfigure}
\usepackage{stackrel}
\usepackage{blindtext}

\usepackage{biblatex}
\addbibresource{library.bib}

\oddsidemargin=0.15in
\evensidemargin=0.15in
\topmargin=-.5in
\textheight=9in
\textwidth=6.25in

\usepackage[colorlinks=true,breaklinks,pdfpagemode=none,linkcolor=blue,citecolor=blue]{hyperref}
\usepackage{enumerate}

%\usepackage{enumitem}
%\setlist{itemsep=0mm}



%\usepackage[usenames,dvipsnames]{pstricks}
%\usepackage{epsfig}
\usepackage{amsmath,amsfonts,amssymb,bm}
%\usepackage{pst-grad} % For gradients
%\usepackage{pst-plot} % For axes


%% Environment definitions (add your own here)

\newtheorem{theorem}{Theorem}
\newtheorem{corollary}[theorem]{Corollary}
\newtheorem{lemma}[theorem]{Lemma}
\newtheorem{observation}[theorem]{Observation}
\newtheorem{proposition}[theorem]{Proposition}
\newtheorem{definition}[theorem]{Definition}
\newtheorem{claim}[theorem]{Claim}
\newtheorem{fact}[theorem]{Fact}

\newenvironment{proof}{\noindent{\bf Proof}\hspace*{1em}}{\qed\bigskip}

%% New commands (add your own here)

\newcommand{\eps}{\varepsilon}
\newcommand{\bbR}{\mathbb{R}}
\newcommand{\hv}{\hat{v}}
\newcommand{\hL}{\hat{L}}
\newcommand{\hlambda}{\hat{\lambda}}
\newcommand{\homega}{\hat{\omega}}
\newcommand{\hp}{\hat{p}}
\newcommand{\hW}{\hat{W}}
\newcommand{\cK}{\mathcal{K}}
\newcommand{\qed}{\rule{7pt}{7pt}}
\newcommand{\cF}{\mathcal{F}}

\begin{document}

   \noindent
   \begin{center}

   \hrulefill
   
   \vspace{5pt}
   
   \makebox[\textwidth]{ {\bf Energy Systems Analysis} \hfill  A.D. Smith 2019}
   \vspace{0pt}
   
   {\Large \hfill  Lecture 2. Review of General Thermodynamics of Energy Systems}
   \vspace{5pt}
   
   \hrulefill
   \end{center}
   
   {\color{darkgray}{\center{ \small{      ``Economies, chemical reactions, ecosystems, and solar systems organize around energy gradients---natural differences in temperature, pressure, and chemistry that set up the conditions for energy flow.''
\\%[3pt]
\rightline{{\rm --- Eric D. Schneider \& Dorion Sagan \cite{schneidersagan}}}}}}}

{\color{darkgray}{\center{ \small{      ``Imperfection (friction, heat leaks, etc.) acts as a brake on the engines (the designs) that drive flow.''
\\%[3pt]
\rightline{{\rm --- Bejan and Zane \cite{Bejan2013-jz}}}}}}}

\section{First Law}

\noindent
\underline{First Law, Closed System:}
%\smallskip

$$\Delta E = \Delta U + \Delta KE + \Delta PE = Q-W$$

\smallskip

\noindent
\underline{Mass Conservation:}
\smallskip

$\dot m= \rho V_{avg}A$ (for uniform $\rho$) \hspace{0.5cm } $\dot V = V_{ave} A$ \hspace{0.5cm } General: $m_{in} - m_{out}= \Delta m_{sys}$

\medskip

\noindent
\underline{First Law, Open System, per unit time:}
%\smallskip

$$\frac{dE}{dt} = \dot Q - \dot W + \sum_{in} \dot m (h + \frac{V^2}{2}+gz) − \sum_{out} \dot m (h + \frac{V^2}{2}+gz)$$

\noindent
Mass and Energy Balances, Steady State (Steady Flow) Process with negligible $\Delta$KE, $\Delta$PE:

$$\sum \dot{m}_{in} = \sum \dot{m}_{out}$$

$$0  = \dot Q - \dot W + \sum \dot{m}_{i}h_{i} -\sum \dot{m}_{e}h_{e}$$

\smallskip

\noindent
\underline{First Law Simplifications}

Closed system, heat transfer at constant volume: \hspace{0.25cm} $\boxed{$$q_{ab}=u_b-u_a}$

Closed system, heat transfer at constant pressure: \hspace{0.25cm} $\boxed{q_{ab}=h_b-h_a}$

Closed system, isentropic compression or expansion: \hspace{0.25cm} $\boxed{w_{ab}=u_a-u_b}$

(\textit{where} $a$ and $b$ represent two states in the cycle.)\\

Open system, heat addition or rejection: \hspace{0.25cm} $\boxed{q=h_e-h_i}$

Open system, isentropic compression or expansion: \hspace{0.25cm} $\boxed{w=h_i-h_e}$

(\textit{where} $i$ and $e$ represent inlet and exit states of a device in the cycle.)\\


\noindent
\underline{First Law, Open System, Transient Process from state 1 to state 2 with negligible $\Delta$KE, $\Delta$PE:}
%\smallskip

$$(m_2u_2 - m_1u_1)_{system} = Q - W + \sum m_ih_i - \sum m_eh_e$$

\subsection{Efficiency}

First Law efficiency is a ratio between the benefit to us of a particular process (such as work) and the cost to us to run that process (such as thermal or chemical energy).\\

\noindent
\underline{Heat Engines:} \\
\noindent
General efficiency: $\eta = \frac{W_{net}}{Q_{in}}$ \hspace{0.5cm} Ideal cycle: $\eta = 1-\frac{Q_{out}}{Q_{in}}$ \hspace{0.5cm} Carnot cycle: $\eta = 1-\frac{T_L}{T_H}$

\medskip

\noindent
\underline{Refrigerators:}\\ 
\noindent
Coefficient of performance:  $COP_R=\frac{Q_L}{W_{in}}$ \hspace{0.25cm} Ideal: $COP_R=\frac{1}{Q_H/Q_L-1}$ \hspace{0.25cm} Carnot: $COP_R=\frac{1}{T_H/T_L-1}$
    
\medskip
    
\noindent
\underline{Heat Pumps:}\\ 
\noindent
Coeff. of performance:  $COP_{HP}=\frac{Q_H}{W_{in}}$ \hspace{0.25cm} Ideal: $COP_{HP}=\frac{1}{1-Q_L/Q_H}$ \hspace{0.25cm} Carnot: $COP_{HP}=\frac{1}{1-T_L/T_H}$


\section{Second Law}

\noindent
Clausius inequality: \oint (\frac{\diff \delta Q}{T}) \leq 0} $\\

%Kelvin-Planck statement: A system cannot produce (net) work while operating on a cycle, in contact with a single thermal energy reservoir.}


\noindent
\underline{Reversible (Ideal) Processes and the Property Entropy}

\noindent
Clausius definition of entropy: $S_2-S_1= \int (\frac{\diff \delta Q}{T})$ \hspace{.25cm} Heat transfer: $Q = \int TdS$ \hspace{0.25cm} per unit mass: $s = \frac{S}{m}$\\

\noindent
\underline{Irreversible (Real) Processes}

 $S_2-S_1} = \int (\frac{\diff \delta Q}{T}) + S_{gen}$ 
\hspace{1.2cm}
$\Delta S_{sys} = S_{in} - S_{out} + S_{gen}$
\hspace{1cm}
Rate form: $\frac{dS_{sys}}{dt} = \dot{S}_{in} - \dot{S}_{out} + \dot{S}_{gen}$
\\

Increase of entropy principle: $\Delta S_{sys,isolated} \geq 0 $\\

\noindent
\underline{Isentropic Efficiencies (Steady-flow Devices)}
\smallskip

\underline{Turbines}
$${\eta}_{t} = \frac{w_a}{w_s} = \frac{h_1-h_{2a}}{h_1-h_{2s}}

\underline{Compressors and Pumps}

\hspace{3.1cm}
${\eta}_{c} = \frac{w_s}{w_a} = \frac{h_{2s}-h_1}{h_{2a}-h_1}$
\hspace{3.3cm}
${\eta}_{p} = \frac{w_s}{w_a} = \frac{h_{2s}-h_1}{h_{2a}-h_1} \approx \frac{v(P_2-P_1)}{w_a}$\\

\noindent
\underline{Entropy Transfer}

$S_{heat} = \int (\frac{\diff \delta Q}{T}) \approx \sum \frac{Q_k}{T_k}$ for each location $k$
\hspace{1.5cm}
$S_{mass}=ms$


\subsection{Efficiency}

Second Law efficiency is a ratio between the performance of a particular device or system and the ideal theoretical performance for the same device or system.

$\eta_{II} =\frac{\eta_{th}}{\eta_{th,rev}}$ \hspace{6cm} $\eta_{II} =\frac{COP}{COP_{rev}}$

\subsection{Entropy generation}

Entropy generation is a quantitative measure indicating how irreversible an actual process is. See Chapter 7 of \cite{cengel}.\\

\noindent
\underline{Entropy Balance}

Closed: 
$S_2 - S_1 = \sum \frac{Q_k}{T_k} + S_{gen} $
\\

Open:
$S_2 - S_1 = \sum \frac{Q_k}{T_k} + \sum m_i s_i - \sum m_e s_e +S_{gen}$
\hspace{.5cm}
Rate:
$\frac{dS_{sys}}{dt} = \sum \frac{\dot{Q}_k}{T_k} + \sum \dot{m}_i s_i - \sum \dot{m}_e s_e + \dot{S}_{gen}$

\subsection{Exergy destruction}

Exergy destruction is proportional to entropy generation. The difference is that it's quantifying the \textit{lost available work}, which is a measure of the reduction in (theoretically) available work due to the condition of the system and its surroundings. See Chapter 8 of \cite{cengel}.

\section{Heat Transfer}

Energy transferred due to a temperature difference falls under the category of heat transfer. I will assume you have familiarity with the basic modes of heat transfer (conduction, convection, and radiation) along with the fundamental equations describing them (Fourier's Law, Newton's Law of Cooling, Stefan-Boltzmann Law). This basic material is summarized in this video from LearnChemE: \url{https://youtu.be/NwOZ1tSe9hs}.

For students who have not yet taken a heat transfer course, or those who want to refresh, I recommend:

\begin{enumerate}
\item Read the Topic of Special Interest section called ``Mechanisms of Heat Transfer'' at the end of the second chapter of our undergraduate thermo text \cite{cengel}. 
\item Familiarize yourself with the idea of view factors as described in this video from LearnChemE: \url{https://youtu.be/UIfRBB49MC4}.
\end{enumerate}

To begin connecting heat transfer concepts with building science, I recommend that you view this video on solar orientation from Solar Schoolhouse: \url{https://youtu.be/OR8EQ0DWpPw}.

\section{Fluids}

Energy and mass transferred due to fluid movement fall under the category of fluid mechanics. I will assume you have familiarity with the basic characteristics of a fluid and are able to formulate a control-volume analysis for a system with fluid flows. This basic material is summarized in this video from LearnChemE: \url{https://youtu.be/-1DVqWmZ9tU}. A simple  control-volume mass conservation problem (solving for flow rate) is solved in this video from 
Water and Wastewater Courses: \url{https://youtu.be/CzyWNamGizA}.

Fluids are used to transfer energy to, from, and within buildings, primarily using just a few common fluid types:

\begin{itemize}[noitemsep]%,nolistsep]
\item air
\item water 
\item refrigerants
\end{itemize}

To begin connecting fluid flow concepts with building science, you will be assigned in the future to watch a video on HVAC systems from our first guest speaker, professional engineer Jane Guyer  (\url{https://youtu.be/Kt1j_aAApwg}).


\section{How thermodynamics affects energy systems analysis}

The conservation laws that express the First Law and the limitations imposed by the Second Law are critical to include in a physics-based analysis. Energy conversion also has implications for the way that humans interact with energy systems, for the financial costs and benefits of energy systems, and for the environmental impacts of energy systems.


% \subsection{People}

% \subsection{Economy}

% \subsection{Environment}


% \section{Understanding thermodynamic performance of energy systems}

% \subsection{Past data (benchmarking)}

% \subsection{Present data}

% \subsection{Future data (prediction)}


% Scn 1.3\cite{Knopf2011-kj}

% license
\bigskip

\noindent
\texttt{\footnotesize RESTRICTED PUBLIC LICENSE --- READ BEFORE SHARING. This is a draft version made available by Amanda D. Smith under a Creative Commons Attribution-NonCommercial-ShareAlike license. 
\href{https://creativecommons.org/licenses/by-nc-sa/4.0/}{CC BY-NC-SA 4.0}}

% references
\newpage
\printbibliography

\end{document}